\subsection{Discussion}

The results of the ICP based approaches depend on the initialisation. This is a disadvantage because estimating the transformation is not straightforward. Even if the initialisation is close to the ideal transformation, these approaches are sensitive to the geometry of the scenes to align. This is also a disadvantage if the scene does not have the good characteristic. However, in case of an ideal scene, these approaches give good results. Coloured ICP helps in the case of flat surfaces. Using the colour of the scene for the alignment of a flat scene provides better results than the point-to-plane ICP and point-to-point ICP.

The global registration procedure is sensible to the geometry of the scene. The different experiments of the section \ref{section:Global registration result} demonstrate that variation in the geometry of the scene helps the RANSAC step finding corresponding points. In the case of a flat scene, the FPFH features are too similar. Therefore, finding relevant corresponding points is not successful. This approach is not robust. However, once the RANSAC step succeeds, the local refinement step depends then on the ICP performances. Point-to-plane ICP was used in these experiments and gave acceptable results. The other variation of ICP could have been applied as well.

The stereo calibration approach is not a success in the presented experiment. Several trials were made. The one presented is the one that gives the best results. The misalignment especially in the X-coordinate could probably be explained by the nature of the image taken to calculate the estimation of the transformation matrix. However, such a method depends on the image taken during the calibration process. It is a non-robust method.

The proposed method is the one that gives the best result, see table \ref{tab:sumary_registration}. The error is below 1 mm in the three different directions. This error is barely visible in a streaming application. Once the ChArUco board point clouds are created, the calculation of the transformation gives always the same result. This is  therefore a robust. 
A drawback is the fact that it is a local approach. It means that the resulting transformation gives good result in the region around the board but further away a misalignment could be noticeable. This method is the one chosen for all the next experiments and steps of this project. 

