
\selectlanguage{english}
\begin{abstract}
\addcontentsline{toc}{section}{Abstract}
\normalsize
    \vspace{4em}
    % Liebherr Machine Bulle company solicits a supply chain design able to deliver the new product without sales forecast within a short lead-time. Indeed the company has to manage more than 50'000 different variants of this hydraulic pump (LH30VO) and therefore ask also a production strategy without planning the end-products. The product launch phase will have an uncertain demand in terms of volume and variants. The company requests some guidance for the ramp-up phase and solutions to minimize the stock value, the stock cost, the fixed costs, and to optimize the number of deliveries to the clients within the time limit. This study proposes a supply chain design for this product.
    % \\
    % \\
    % This analysis is interesting as it simulates the production flow to propose the best supply chain design with a short lead time. This study proposes an adequate production strategy for a highly customized product by running multiple simulations with different parameters. Uncertainty on demand is simulated by a random selection of product variants and quantity ordered at each simulation. This approach determines the best strategy whatever the possible scenario.
    % \\
    % \\  
    % This master thesis uses a simulation-model to find out the best design. First, the current flow of operations is modeled, tested, then validated. Secondly, the model of operation is reused to design different possible supply chain strategies. The created models are run to get the results that are compared later.
    % \\
    % \\
    % The discussion of the results leads to the following solutions:  
    % \vspace{0.25em}
    % \begin{itemize}
    %     \item To optimize the fixed costs, the assemble-to-order strategy is advised.
    %     \item The make-to-order strategy has the lowest stock costs.
    %     \item To optimize the quantity of on-time deliveries during the ramp-up phase:
    %         \begin{itemize}
    %             \item The assemble-to-order strategy has the best results at the beginning.
    %             \item The make-to-order strategy has the best results at the end.
    %         \end{itemize}
    %     \item A full pull strategy minimizes the inventory value.
    %     \item The following mix strategy has the best results based on a complete comparison.
    %         \begin{itemize}
    %             \item The assemble-to-order strategy for the beginning of the ramp-up.
    %             \item The make-to-order strategy since June 2021.
    %         \end{itemize}
    % \end{itemize}
    % \vspace{1em}
    % The current processes of the company are not suitable for a supply chain driven by consumption. Therefore a road-map is discussed and proposed to conduct the implementation of the proposed supply chain design, with firstly changes on the inventory management by implementing an economic order policy. Secondly, with changes in production planning by implementing the order point strategy. 
    % \vspace{1em}
    % \\
    % Area of improvements impacting the supply chain such as:
    % \vspace{0.25em}
    % \begin{itemize}
    %     \item Digitalization
    %     \item Forecast methods
    %     \item Process standardization
    %     \item Inventory management
    %     \item Product development
    % \end{itemize}
    % \vspace{0.5em}
    % are presented to pave the way for future projects.
    
    
    Having the possibility to change freely the point of view in a video stream is the purpose of this master project. As connected meeting rooms are a current key market of the Logitech business, adding this feature to their products could help the business to grow. Developing such technology for meeting rooms could increase the immersion feeling of the different participants. The aim of this project is to propose a pipeline to render 3D video data of the connected conference rooms. To achieve this objective, RGB-D Azure Kinect cameras are used. The RGB and depth streams of these cameras are fused to create point clouds. This 3D representation of the scene enables free viewpoint possibilities. It could make the online user experience closer to the one sitting in the room. Several cameras have to be used to cover enough volume of the room. Two cameras are used in this project as a proof of concept. Using a multi-camera set-up results in several problems to solve. The different point clouds created have to be aligned. For this purpose, a registration approach is proposed. Aligning several views results in an inconsistency in the colour which is partially handled by an algorithm developed in this project. Because the depth information of the Kinect camera is not stable enough, the resulting created point cloud flickers. An algorithm based on a moving average filter is presented to tackle this issue. The depth and RGB streams are not always perfectly aligned, therefore the resulting point cloud has some noisy edge effects. This work proposes a method to soften this edge effect. The visual enhancement algorithms proposed in this report does not resolve all the problems. However, the resulting output video stream is acceptable to be used in the targeted scenario. Some use cases are proposed to make the user experience more immersive. The proof of concept of the free point of view is, therefore, demonstrated but further work should be done on the topic to improve the whole pipeline in order to have a deliverable product.
    
  
\end{abstract}
\clearpage


\selectlanguage{english}
