\subsection{Thesis structure}
\label{section:thesis_structure}

% The master thesis is structured as follows; Section \ref{section:context} provides the context of the project.
% Section \ref{section:literature_review} examines the literature done on the subject and what is missing.
% Section \ref{section:proposed_method} proposes methods to answers the research questions. The first section \ref{section:proposed_model_supply_chain} proposes a simulation-model of the current supply chain. The second section \ref{section:proposed_model_supply_chain_ramp-up} proposes models of the different possible supply chain strategies during a ramp-up phase.
% Section \ref{section:result} presents the results issued from the proposed models.
% Section \ref{section:discussion} discuss the results and reviews the proposed models.
% Section \ref{section:conclusion} concludes the master thesis, brings to the fore the findings, and suggests propositions for
% future research.

The master thesis is structured as follow: Section \ref{section:3d scene representation} examines the literature relating to the subject. Section \ref{section:pipeline} proposes a pipeline for 3D video rendering of a scene. Section \ref{section:apparatus} presents the hardware used during the project. Section \ref{section:Registration} compares different registration approaches and proposes a new one. Section \ref{section:Visual enhancement} presents different algorithms that improve the visual rendering of the scene. Section \ref{section:general_discussion} discusses the different results and reviews the proposed pipeline. Section \ref{section:Use case scenarios} presents different potential applications and scenarios for Logitech. Section \ref{section:conclusion} concludes the master thesis and proposes future research perspectives.